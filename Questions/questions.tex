\documentclass{article}
\usepackage{graphicx}

\begin{document}

\title{Questions and Notes}
\author{Victor \& Mikkel}

\maketitle

\section{Questions}
\begin{itemize}
	\item train\_data\_checkpoint.json:\\
	Trained on wikipedia and articles for what purpose?
	To understand "security", "privacy", etc.?
	When is this used?
	\item Commercial intentions?:\\
	Is the intention to use this commercially?
	The API specifically states that this should not be used commercially.
	If commercial aspects are of interest, we need to have it approved.


\end{itemize}

\section{Notes}
\begin{itemize}
	\item Reddit API:\\
	client ID: fQuRCESlv\_7yOQ\\
	client secret: v6AjzytAKDgqjqyQNczapezXUjk
\end{itemize}

\section{Overview}
\begin{itemize}
	\item reddit\_API.py:\\
	Uses the Reddit API to gather top tweets
	from the "smarthome" subreddit and comments.
	Now set up to use Victor's account.\\
	\emph{Questions}: \\ Why does it not save them?
	How many tweets can/should we get?
	\item scrape.py:\\
	Generates a user-agent to access the different
	articles used for generating the \textit{train\_data.json} file.\\
	\emph{Questions}: \\
	Why are there so many functions for scraping?
	Is it because they differ in format and need different preprocessing to read?
	\item pushshift\_aggs.py:\\
	Used to scrape the frequency of query-terms
	\textit{"privacy", "security", "trust"} for the last 3 years on the subreddits.
	It seems like the \textit{frequency} keyword establishes the sampling rate.
	This is used to plot how many comments uses these query words over 3 years.
	\emph{Questions}: \\
	Is this actually the frequency?
	\item pushshift\_API.py:\\
	Used to scrape threads from two subreddits.
	(homeautomation, smarthome) based on three
	query terms (privacy, security, trust). If
	a comment (not post?) matches one of these
	terms then the whole thread is downloaded.
\end{itemize}

\section{Meeting}
\begin{itemize}
	\item Questionaire:
	\item How far back: as far as possible.
	\item Quantitative survey: trust, privacy, security (not really important). Coming into the frame when people have issues. Only important when something happens. First general idea (API for general, push-shift for general). Papers/Wikipedia etc. to train on concepts (privacy, security, trust). For ELMO to learn the concepts.
	\item Further: assemblage theory: human-object vs. human-human relationships. (master-servant, friend). Do they use it more or less when they have different relations. Interest in outcome. Do these relationships exist? Linguistic features of master-servant vs. partner-partner. Does this predict whether people expand to more SmartHome - BigFive as well. Low-hanging fruit (sentiment analysis).
\end{itemize}

\end{document}
